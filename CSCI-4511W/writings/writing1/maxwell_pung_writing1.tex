% Author: Maxwell Pung
% Class: CSCI4511W
% Date: 1.23.17
% Writing Assignment 1

\documentclass[12pt]{article}

\begin{document}

\title{Concerning Important Issues of AI Technologies}
\author{Maxwell Pung\\
	CSCI4511W Artificial Intelligence,\\
	Minneapolis,\\
	Minnesota}
\date{\today}
\maketitle

\newpage

\begin{abstract}
This paper addresses what I consider to be the two most important issues that must be addressed if we are to use Artificial Intellgience (AI) technologies such that all of society is rationally benefited. The first issue that must be addressed is removing biases in the data that allows AI technologies to make informed decisions. The second issue that must be addressed is ensuring that AI technologies are fairly distributed across society otherwise existing inequalities could expand. If these issues are not addressed, AI technologies have the potential to cause more harm than good for society. 
\end{abstract}

\paragraph{Addressing Bias In AI Data\\}
AI technologies are driven by massive amounts of data that calibrate machine learning algorithms and dictate how the model of the world they create is constructed. This means that if the data being used to teaach machines holds the characteristic of being biased, then that bias will be reflected in decisions made by the AI technology. The influence of bias in machine learning algorithms could create unfair concentrations of those that receive the benefits from AI. AI technologies will have the greatest positive impact for society if bias is eliminated from the data used in machine learning algorithms, otherwise the technology will not be able to used to fix important problems in domains that AI is applicable to. It is therefore imperitive that discrimination based on things like race and sexual orientation is avoided. If bias is eliminated from machine learning algorithms, AI would become an extremely useful tool in making decisions like who should receive loans and who should go to jail \cite[pg. 43]{twentythirty}. Such decisions rely on unbiased decision making in order for justice to genuinly be preveserved. Flushing bias from AI decisions is important because AI has the potential to detect, remove, and reduce human bias in decision making \cite[pg. 37]{twentythirty}. This would be an incredible step towarding ensuring equal opportunities for everybody. 

\paragraph{Fair Distribution of AI Technologies\\}
The opportunities that will arise with the advancement of AI... It will be important for us as a society to ask ourselves who is benefiting from such technologies and take action if the answer does not resonate with us. Taking such actions requires people to be informed and being informed requires access to AI technologies. AI technologies must be fairly distributed across society such that they do not expand existing inequalities of opportunity in areas such as income and access to resources \cite[pg. 10]{twentythirty}. AI technology creates opportunity for higher wages and more opportunity for leisure for workers \cite[pg. 2]{gov}. Policies and choices made affecting the governance of AI will greatly influence if all get the chance to capitalize on such opportunities. If the distribution of AI technologies allowed for everybody to have the opportunity to enjoy the economic benefits of AI, I would define the distribution as precisely fair. 

\paragraph{What Can Be Done\\}
Government policies and regulations will play a critical role in how far the success of AI reaches. As of now, those making decisions in governing bodies do not possess the expertise to regulate such technologies. It is important that the government create policies that encourage competition and support competition in the market \cite{gov}. Without competition in the AI market, access to AI technologies for common folk could become for difficult. Any attempts made by the government to regulate AI in a general manner will be misguided becauses the risks involved differ by the domain the technology is applied to \cite[pg. 48]{twentythirty}. Such domains include machine learning applications, natural language processing, computer vision, and smart robots. The government must balance incentivizing AI innovation while protecting upcoming technologies from third party harm at the same time \cite[pg. 47]{twentythirty}. Such a balance would maximize opportunity for access to AI educational resources and help distribute the benefits of AI fairly across society. If no protection is in place, it will become more difficult for upcoming AI technologies to become commonplace. Protection upcoming technolgoies goes hand in hand with encouraging healthy competition. The government must also consider reworking existing policies such as the Computer Fraud and Abuse Act (CFAA) and the Digital Millennium Copyright Act (DMCA). The CFAA and DMCA legislation threaten to restrict research into AI accountability and fairness; it should be clarified in these pieces of legislation that they are not intended to restrict such research \cite[pg. 4]{now}. Such legislation could directly impact AI research negatively which would also interfere with the fair distribution of AI technologies and access to them. The government and institutions must support research to measure the accuracy and fairness of AI. Without this kind of research, AI will not be able to reach it's full potential as a technology that could have a widespread, positive impact on society. 

In order for AI to reach it's full potential, bias in the data serving as input to machine learning must be removed. One solution to the problem is to prioritize diversity and inclusion in both STEM and AI fields \cite{gov}. Although the proposed solution does not tackle the issue directly, it is a step in the right direction. The more diverse the developers of AI technologies are, the more diverse the technology is going to be itself. After all, there is nobody better at discovering and identifying bias than those that have lived it. Another way that we can begin weeding out bias of AI data is to promote interdisiplinary research initiatives \cite[pg. 5]{now}. Similarly to the first proposed solution, this solution would promote diversity in AI research projects and therefore promote less bias in AI data. The previously mentioned solutions do not address bias in AI data using a technical approach. An example of a technical approach to removing gender bias in AI data has been proposed and is known as "re-training" the data. Researchers provided a methodology for modifying data to remove stereotypes such as receptionist and female while keeping desired associations like queen and female in-tact \cite{debiasing}. 

\newpage

% syntax: \cite{name} 
\bibliography{writing1}{}
\bibliographystyle{ieeetr}

\end{document}
